\documentclass[12pt,a4paper]{article}
\usepackage[latin1]{inputenc}
\usepackage[german]{babel}
\usepackage{amsmath}
\usepackage{amssymb}
\usepackage{amsfonts}
\usepackage{amsthm}

\setlength{\oddsidemargin}{-0.5cm}
\setlength{\evensidemargin}{-1cm}
\setlength{\textwidth}{17cm}
\setlength{\textheight}{25cm}
\setlength{\headsep}{1cm}
\setlength{\topmargin}{-1.5cm}
\setlength{\parindent}{0cm}

\begin{document}

\theoremstyle{plain}  \newtheorem{satz*}{Satz}[]

\section*{Quicksort}

Quicksort ist ein allgemeines Sortierverfahren vom Typ "<Teile und Herrsche">. 
Es beruht auf dem Zerlegen eines zu sortierenden Feldes in zwei Teile und dem anschlie�enden 
Sortieren der beiden Teile unabh�ngig voneinander. Der Algorithmus hat folgende Gestalt:

\begin{verbatim}
void quicksort (int A[], int l, int r)
{
  if(l >= r) return;  
  int i = l;
  int j = r+1;
  int v = A[l];
  for (;;)
  {
    while(A[++i] < v && i < r);
    while(A[--j] > v);
    if(i >= j) {
      swap(A, l, j);    			
      break;
    }
    swap(A, i, j);
  }
  quicksort(A, l, j-1);
  quicksort(A, j+1, r);
}
\end{verbatim}
\mbox{}\\[-1ex]
Die Parameter {\tt l} und {\tt r} begrenzen das Teilfeld innerhalb des urspr�nglichen Feldes, 
welches zu sortieren ist; der Aufruf {\tt quicksort(A, 0, A.size()-1)} sortiert das gesamte Feld. 
\\\\[-1ex]
Das entscheidene Element der Methode ist der Programmcode innerhalb der unendlich Schleife 
{\tt for(;;)}, der das Feld so umordnet, da� die folgenden Bedingungen erf�llt sind:

\begin{enumerate}
\item [i.]   F�r ein beliebiges {\tt j} befindet sich das Element {\tt A[j]} an seinem 
					   endg�ltigen Platz im Feld.
\item [ii.]  Alle Elemente {\tt A[l],\ldots,A[j-1]} sind kleiner oder gleich {\tt A[j]}.
\item [iii.] Alle Elemente {\tt A[j+1],\ldots,A[r]} sind gr��er oder gleich {\tt A[j]}.\\[-1ex]
\end{enumerate}

Im ersten Schritt wird das zerlegende Element {\tt v = A[l]} ausgew�hlt,
das in seine endg�ltige Position gebracht werden soll. Jetzt beginnt die Suche von links, 
bis ein Element gefunden wurde, welches gr��er als {\tt v} ist, anschlie�end startet die Suche
von rechts, bis ein Element gefunden wurde, das kleiner als {\tt v} ist. \\
Haben sich die Zeiger {\tt i} und {\tt j} noch nicht getroffen, dann werden die Elemente  
{\tt A[i]} und {\tt A[j]} miteinander vertauscht, andernfalls wird durch einen weiteren Austausch
die Beding- \linebreak ung i. sichergestellt und die unendlich Schleife durch ein {\tt break} verlassen. \\

Im folgenden wird die Funktion f�r das linke Teilfeld und danach f�r das rechte Teilfeld
rekursiv aufgerufen. Am Ende ist das Feld aufsteigend sortiert.\\

\newpage 

{\bf Korrektheit des Vefahrens}
\begin{proof}
\mbox{}\\
Zu zeigen ist, da� in jedem Iterationsschritt gilt:\\[-2ex]
\begin{center}
{\tt A[l],\ldots A[i] $\leq$ A[j],\ldots,A[r]}.
\end{center}
\mbox{}\\
{\em Induktionsanfang} \\[-2ex]

Vor der ersten Interation ist {\tt i = l} und {\tt j = r+1}.\\

{\em Induktionsschritt} \\[-2ex]

D.h. der �bergang von {\tt i}$_{old} \to $ {\tt i}$_{new}$ und {\tt j}$_{old} \to $ {\tt j}$_{new}$.\\

Nach Induktionsannahme gilt:\\[-2ex]
\begin{center}
{\tt A[l],\ldots,A[i$_{old}$] $\leq$ A[j$_{old}$],\ldots,A[r]}
\end{center}
\mbox{}\\
Nach Konstruktion gilt:
\begin{center}
{\tt A[i$_{old}$+1],\ldots,A[i$_{new}$-1] $\leq$ A[j$_{new}$+1],\ldots,A[j$_{old}$+1]}
\end{center}
hieraus folgt
\begin{center}
{\tt A[l],\ldots,A[i$_{new}$-1] $\leq$ A[j$_{new}$+1],\ldots,A[r]}
\end{center}
weiter gilt:
\begin{center}
{\tt A[i$_{new}$] $\geq x $} und {\tt A[j$_{new}$] $\leq x $}
\end{center}
\mbox{}\\
Nach Vertauschen von {\tt A[i$_{new}$]} mit {\tt A[j$_{new}$]} gilt:\\[-2ex]
\begin{center}
{\tt A[l],\ldots,A[i$_{new}$] $\leq$ A[j$_{new}$],\ldots,A[r]}
\end{center}
\end{proof}
\mbox{}\\

{\bf Laufzeitverhalten (worst-case)}\\[-1ex]

Die Vermutung ist, da� ein schlimmster Fall eintritt, wenn Quicksort eine aufsteigend sortierte Liste
sortieren soll. Dann f�llt n�mlich in jedem Rekursionsaufruf immer nur ein Element aus der Liste der noch 
zu sortierenden Folge (siehe Animation). Die resultierende Laufzeit $T(n)$ hat die Komplexit�t

\begin{align*}
T(n)&=\text{$\cal O$}\;\left(\sum_{i=2}^n i + n \cdot c\right)\\
    &=\text{$\cal O$}\;\left(n^2\right)\\
\end{align*}

\newpage

\begin{proof}
\mbox{}\\
Allgemein gilt:
\begin{align*}
T(n) &=\max_{1\leq q \leq n-1} \{ T(q) + T(n-q) + c_1 \cdot n\}
\intertext{Es wird nun gezeit, da� $T(n) \leq \left( c\cdot n^2\right)$ f�r $c\geq 
\dfrac{c_1}{2}\cdot\dfrac{n_0}{n-1}$ f�r $n_0$ gro� genug gilt.}
T(n) &\leq\max_{1\leq q \leq n-1}\{T(q) + T(n-q) + c_1\cdot n\}\\
     &\overset{\text{I.A.}}{\leq} \max_{1\leq q \leq n-1}\{c\cdot q^2 + c\cdot (n-q)^2 + c_1\cdot n\}\\
     &= c \cdot \max_{1\leq q \leq n-1}\{ q + (n-q)^2 - 2q \cdot (n-q) \} + c_1 \cdot n\\
     &= c \cdot n^2 - 2c \cdot \min_{1\leq q\leq n-1}\{q\cdot (n-q)\} + c_1\cdot n\\
     &= c \cdot n^2 - 2c \cdot 1(n-1) + c_1 \cdot n\\
\intertext{f�r $n$ gro� genug folgt}
T(n) &= c\cdot n - 2\cdot \frac{c_1}{2}\cdot\frac{n}{n-1} \cdot (n-1) + c_1\cdot n\\
     &= c \cdot n^2
\end{align*}
\end{proof}
\mbox{}\\
In der Praxis kommt der Algorithmus sehr oft zum Einsatz, dies liegt unter anderem an der
Laufzeit im Mittel, die im folgenden betrachtet wird.\\

\begin{satz*}
Quicksort ben�tigt im Mittel $2\log n + \Theta(1)$ viele Vergleiche. Durch randomisiertes
Quicksort (zuf�lliges Vertauschen von zwei Elementen in dem noch zu sortierenden Feld) wird 
der schlechteste Fall mit hoher Wahrscheinlichkeit vermieden.\\
\end{satz*}

\begin{proof}
\begin{align*}
QSA(n) = n + 2 + \frac{1}{n-1}\cdot&\sum_{1\leq q \leq n - 1} (QSA(q) + QSA(n-q)) 
\intertext{mit}
\sum_{1\leq q \leq n - 1} QSA(q) &=  \sum_{1\leq q \leq n - 1} QSA(n-q))
\intertext{folgt}
QSA(n) &= n + 2 + \frac{2}{n-1}\cdot\sum_{1\leq q \leq n - 1} QSA(q)\\\\
(n-1)\cdot QSA(n) &= (n+2)\cdot(n-1) + 2 \cdot \sum_{1\leq q \leq n - 1} QSA(q) \qquad |- \\
(n-2)\cdot QSA(n) &= (n+1)\cdot(n-2) + 2 \cdot \sum_{1\leq q \leq n - 1} QSA(q) \qquad  \\\\
(n-1)\cdot QSA(n) - (n-2)\cdot QSA(n-1) &= \underbrace{(n+2)\cdot(n-1)}_{n^2-n + 2} 
- \underbrace{(n+1)\cdot(n-2)}_{n^2+n - 2} + 2\cdot QSA(n-1)\\\\
(n-1)\cdot QSA(n) &= 2\cdot n + n \cdot QSA(n-1) \qquad |: n\cdot(n-1)\\\\
\frac{QSA(n)}{n} &= \frac{QSA(n-1)}{n-1} + \frac{2}{n-1}\\
&= \frac{QSA(n-2)}{n-2} + \frac{2}{n-2} + \frac{2}{n-1}\\
&= \frac{QSA(n-2)}{n-3} + \frac{2}{n-3} + \frac{2}{n-2} + \frac{2}{n-1}\\
& \ldots\\\\
\intertext{$QSA(k) = c$ f�r ein festes $k$ folgt}
\frac{QSA(n)}{n} &= \Theta\left( c + 2 \cdot \sum_{i=1}^{n-1} \frac{1}{i}\right) \\
                 &= 2 \log n + \Theta(1)\\
                 &= \Theta\left(\log n\right)\\\\
QSA(n) &= \Theta\left(n \log n\right)
\end{align*}
\end{proof}

\end{document}
