\documentclass[12pt,a4paper]{article}

\usepackage{amstex}             % AMS extra math features

%\usepackage{umlaut}             % German umlauts
%\input{a4}                      % Adjust text and margin sizes

\renewcommand{\l}    {\textsc{Leda}}
\newcommand  {\ps}   {\textsc{PostScript}}
\newcommand  {\psf}  {\pr{ps\_file}}
\newcommand  {\pr}[1]{\textbf{\texttt{#1}}}
\newcommand  {\tab}  {\hspace*{4cm}}
\newcommand  {\wi}   {12cm}
\newcommand  {\dt}   {0.5cm}
\newcommand  {\tp}[1]{\makebox[2.4cm][l]{\pr{#1}}\pr{F.}}

\begin{document}
\vspace*{0.5cm}
\begin{center}
\huge
User Manual\\
for the\\
\l-to-\ps\\
Converter\footnote{This work was financially supported by the Deutsche
  Forschungsgemeinschaft (DFG), Proj.-Nr. No 88/14-1.}\\
\vspace{1cm}
\Large
Thomas Wahl\\
Lehrstuhl f�r Informatik I\\
Universit�t W�rzburg\\
\today\\
\end{center}

\vspace{1.5cm}
\tableofcontents

\vfill
\noindent
This subdivision agrees with the one in Chapter 10 - Graphics - in the
\l\ manual\footnote{The \l\ User Manual, Version R 3.3, by Stefan
  N�her (University of Halle) and Christian Uhrig (Max Planck
  Institute for Computer Science, Saarbr�cken), Germany} (sections
10.1 and 10.2)
%
\newpage
%
\part*{Introduction}
\addcontentsline{toc}{part}{Introduction}
%
This manual describes a \l-to-\ps\ converter, which enables programmers
using the graphics and windows facilities of the \l\ library to redirect
graphical output, as points, lines, circles, polygons, and text
representations, from the screen window into a simple \ps\ file. This
file may then be interpreted by a \ps\ previewer, like
\textsc{GhostView}, it may be printed or be processed otherwise
like any \ps\ source.

The converter is written in C++ and was compiled by the \textsc{Gnu} 2.7
compiler. It not only converts, but also \textit{uses} features of the
\l\ library; so the entire \l\ package should be available when using
the converter.

The converter consists of two files: an object file and a C++-header
file. The latter should be placed in your file system where all \l\
header files are located; e.g. in /usr/local/lib/g++-include/LEDA. It
can then be included into a C++-program like any other \l\ header
file, for example \pr{\#include <LEDA/ps\_file.h>}. --- The object
file must be linked to the object source of your program, for instance,
if you make a library archive libpsf.a from ps\_file.o, you can
use the -l option with argument \pr{psf}:\\
\pr{g++ \ldots -lpsf \ldots}.

\vspace{\dt}
The main emphasis while writing the converter was to force the user to
make as few changes as possible when redirecting the output from the
screen to a \ps\ file. In the current version, the user only has to
change the name of the constructor (\pr{window} becomes \psf)
and, if so desired, add some parameters. Further, the \pr{display}
method has to be removed, as it is not needed (nothing is displayed,
but written to a file). All other implementations, as the
initialization (\pr{init}) and (almost) all drawing operations, are to
be used in exactly the same way as one is used to in connection with
the \pr{window} type. If the resulting \ps\ file is viewed, the output
should exactly look like the one created by a \pr{window} on the
screen. Exceptions to this rule are the \pr{clear} method (cf. 2.3.1)
and some few operations that have not (yet) been implemented.

There are, however, some enhancements to windows, which are in
most cases due to sophisticated \ps\ features. Some of those
enhancements are:
%
\begin{itemize}
\item The output can be split into several pages and into several
  files (cf. \pr{newpage} and \pr{newfile} in 2.3.1)
\item The range of the coordinates (\emph{xmin}, \emph{xmax}, \emph{ymin})
  can be changed at any time for one and the same \psf\ object
  without clearing the output produced so far. This may be useful
  along with the splitting option above (cf. \pr{init} in 2.3.1)
\item The constructor accepts size specifications in cm for a box that
  the output is to be placed into (cf. Creation in 2.2)
\item When working with a non-color printer or display, one can assign
  an arbitrary gray value to any color in which this color should be
  displayed. Thus, important markings in glaring colors (on a color
  display) can be drawn in, say, light gray values on a monochrome
  screen or printer. (cf. Colors in section 1)
\end{itemize}

The header file contains a number of symbolic constants, some of which
influence the behavior of the converter by means of numerical
values. It might be desired to alter these values in adapting the
package to local circumstances.

\noindent
Here the meaning of some of the constants:
\begin{description}
\item [\pr{VERSION}] the name of the \ps\ version that the converter
  translates into. This string appears in the header of the \ps\ file
  for reasons of conformity.
\item [\pr{PIXEL}] According to the \ps\ Handbuch\footnote{POSTSCRIPT
    HANDBUCH, 2. �berarbeitete Auflage 1989, Addison-Wesley
    (Deutschland) GmbH}, a typographical dot has a diameter of about
  1/72 inches, therefore 2.54/72.0 -- the value of this constant --
  represents the size of a dot in cm. This value is needed for scaling
  purposes and should not be changed.
\item [\pr{X\_SHIFT, Y\_SHIFT}] the number of pixels that the entire
  picture is shifted from the lower left corner to place it somewhat
  in the middle of the \ps\ window.
\item [\pr{HEIGHT, WIDTH}] the default height and width in cm of the
  box in which to place the output. These values apply if no height or
  width is specified by the user in the constructor, and it
  corresponds to the size of the drawing area on an A4-paper format.
\item [\pr{SIZE}] the default coordinate range. As long as no
  \pr{init} method is called for a \psf\ object, the triple
  (\emph{xmin}, \emph{xmax}, \emph{ymin}) is assumed to be
  (0,\pr{SIZE},0).
\end{description}
%
\section{Colors (\pr{color})}
%
The data type \pr{color} can be used within the \psf\ class in the same
way as in connection with the \pr{window} class, i.e., the \pr{color}
constructor and all operations manipulating rgb-values of a color
are available. Also, all drawing operations have an optional color
argument for temporary switching the foreground color to a new value,
restricted to the current drawing operation. As mentioned in the \l\
manual (section 10.1 Colors), there are 16 predefined colors, which are
available in \psf\ applications also.

\vspace{\dt}
According to the \l\ manual, when working with a monochrome display,
all colors different from \pr{white} are represented as
\pr{black}. This might not be the method of choice, especially when it
is desired to make a print-out of the drawing, which is possible in
connection with the \psf\ class.

The \ps\ language offers a way to represent any color in gray
values, i.e. a number between 0 and 1 that determines the share of
white and black in the representation. This method was adopted for the
\psf\ class such that, on monochrome displays, colors are
automatically reduced to those gray values, but not just to black and
white.

Further, even on a color display, it might be desired to show
only gray values instead of colors, for example when the user prepares
a black-and-white print-out on a color display. For this purpose, it
is possible to switch within one and the same \psf\ file object between
color and gray value representation of the drawings. This is
accomplished by the \pr{set\_output\_mode(output\_mode new\_mode)}
function, which accepts \pr{colored\_mode} and \pr{gray\_mode} as
arguments. (cf. Setting parameters in 2.3.2)

\vspace{\dt}
In some cases, however, the gray value chosen by the \ps\ language
may not be the one prefered by the user. For example, red usually gets
assigned a dark gray value, whereas green is represented in a light
hue. That seems unreasonable if the user wants the red markings to
show important features of the drawing.

Therefore, one can associate such a gray value to a color oneself:\\
The \pr{set\_gray\_value(color c, double d)} function takes as the
first argument the color the gray value of which is to be
re-assigned. The second argument is the gray value itself and ranges
from 0 (black) through 1 (white).

\vspace{\dt}
\noindent
\textbf{Implementation:}\\
Gray values for colors are stored in a list of element type \pr{pair}, where
\pr{pair} is a struct with a \pr{color} and a \pr{double} attribute
for the gray value.
%
\section{\ps\ files (\psf)}
%
\subsection{Definition}
%
The date type \psf\ is a graphical input/output interface for the
familiar \l\ drawing operations of two-dimensional geometry. Unlike
the data type \pr{window}, the output produced by a \psf\ object is
\emph{permanent}, i.e., it is not lost after exiting the C++-program as
it is saved in an output file.

\vspace{\dt}
An instance of type \psf\ is (as far as the user takes notice of it)
an ordinary ASCII file that contains the source code of the graphics
output in the \ps\ description language. After running the
C++-program, the file is created in the user's current working
directory and can later be handled like any other \ps\ file, i.e., it
may be viewed, printed etc.

Of course, features like a panel section (as in \pr{window} type
instances) don't make sense for a representation that is not supposed
to be displayed on the screen and interactively worked with by the
user. Therefore, only drawing operations are applicable to a \psf\
instance.

Determining the \emph{xmin}, \emph{xmax}, \emph{ymin}, and \emph{ymax}
parameters follows the same philosophy as in the \l\ library, so see
the \l\ manual and the \pr{init} operation in this manual.

\vspace{\dt}
\noindent
A list of important (to the user) \psf\ parameters:
%
\begin{enumerate}
\item The \emph{xmin}, \emph{xmax}, \emph{ymin}, and \emph{ymax}
  parameters determine the coordinate range of the drawing area which
  should be kept to by the user. These parameters can only be changed
  using the \pr{init} method.
\item The \emph{draw bounding box} parameter determines whether the
  box that the output is placed into is supposed to be \emph{drawn} or
  not. However, the bounding box will be \emph{defined} in any
  case. (cf. \pr{HEIGHT} and \pr{WIDTH} symbolic constants in the
  Introduction)
\item The \emph{output mode} parameter determines whether colors or
  gray values are to be used for drawing objects. (cf. Colors in
  section 1)
\item The \emph{grayvalue} array parameter contains gray values for up
  to \pr{MAXDEFCOL} colors which are to be used if the \emph{output
  mode} parameter is currently \pr{gray\_mode}. (cf. Colors in section 1)
\item The \emph{foreground color} parameter defines the default color
  of all drawing operations if no drawing color is specified in the
  optional color argument.
\item The \emph{linewidth} parameter (in ``units'' [see next])
\item The \emph{unitlinewidth} parameter determines the width of a
  line \emph{in cm} with \emph{linewidth} $\mbox{parameter}=1$. If the
  current linewidth is 2 (units) and unitlinewidth is 0.5 (cm), all
  lines are 1cm wide.
\item The \emph{linestyle} parameter
\item The \emph{nodewidth} parameter (in pixels)
\item The \emph{crosswidth} parameter defines the width of a cross
  \emph{in cm} representing a \emph{point}. (cf. \l\ manual,
  \pr{draw\_point} operation)
\item The \emph{arrowlength} parameter defines the length \emph{in cm}
  of the two side branches of an arrowhead (\pr{draw\_arrow}
  operation).
\item The \emph{arrowangle} parameter defines the angle between the
  two side branches of an arrowhead and the arrowline.
\item The \emph{textfont} parameter defines the font to be used in all
  text drawing operations.
\item The \emph{fontsize} parameter defines the height of text
  representations \emph{in cm}.
\end{enumerate}
Where no explanation is given, consult the \l\ manual.

\vspace*{\dt}
\noindent
The following parameters of the \pr{window} type are \emph{not}
available for the \psf\ data type:
\begin{enumerate}
\item The \emph{background color} parameter
\item The \emph{gridwidth}        parameter
\item The \emph{frame label}      parameter
\item The \emph{show coordinates} parameter
\item The \emph{flush output}     parameter
\item The \emph{textmode}         parameter
\item The \emph{drawing mode}     parameter
\item The \emph{redraw function}  parameter
\item The \emph{buttons per line} parameter
\end{enumerate}
The functions that are associated with these parameters either have to
be implemented yet (e.g., \emph{textmode}), or they don't make sense
in this particular application (e.g., \emph{flush output}). Whenever
it is attempted to call a function that is not implemented, a warning
is issued both on standard error output (screen) and in the \ps\ file
that the output is written into.
%
\subsection{Creation}
%
\psf\ \pr{F(double w, double h=HEIGHT, string name="unnamed.ps")}\\
\tab
\parbox[t]{\wi}{creates a \psf\ object with a file of name \pr{name}
  associated to it. Instead of \pr{string}, the type of \pr{name} may
  also be \pr{char *}.

  The picture represented by the \ps\ file will have physical size
  \pr{w} $\times$ \pr{h} cm (not pixels!)}
\\[\dt]
\psf\ \pr{F(string name="unnamed.ps")}\\
\tab
\parbox[t]{\wi}{same as \psf\ \pr{F(WIDTH,HEIGHT,name)}}
\\[\dt]
Remember that the \pr{WIDTH} and \pr{HEIGHT} symbolic constants stand
for A4 paper format (16.0 $\times$ 23.5 cm drawing area).

All constructors initialize the coordinate range of \pr{F} to
\emph{xmin} = 0, \emph{xmax} = 100, and \emph{ymin} = 0, where 100
is the value of the \pr{SIZE} symbolic constant (see
Introduction). The \emph{ymax} value depends on \pr{w} and \pr{h}.
%
\subsection{Operations}
%
\subsubsection{Initialization}
%
\pr{void} \pr{F.init(double x0, double x1, double y0)}\\
\tab
\parbox[t]{\wi}{sets \emph{xmin} = \pr{x0}, \emph{xmax} = \pr{x1}, \emph{ymin} =
  \pr{y0}, and \emph{ymax} such that $\Delta y/\Delta x = h/w$. $w$ and $h$ are
  width and height of the surrounding box. All \psf\ parameter values,
  like foreground color, linewidth etc., survive after the \pr{init}
  operation.}
\\[\dt]
\pr{void} \pr{F.clear()}\\
\tab
\parbox[t]{\wi}{differs slightly from \pr{window::clear()}:
  Instead of clearing the file, it opens a new (empty) page,
  whereas the output produced so far is maintained on previous
  pages. All \psf\ parameter values survive at the new page also.
  (\pr{window::clear()} also keeps all settings after clearing the
  page.)

  This behavior of \pr{clear} implies that -- unlike
  \pr{window::clear()} -- a call of this method \emph{before} any
  drawing operation has been carried out, i.e. immediately after
  creating the \psf\ object, is \emph{not} necessary and not
  reasonable either. The result would be an empty page at the
  beginning of the \ps\ output.}
\tab
\parbox[t]{\wi}{Another difference to the corresponding \pr{window}
  method is that it does not have an optional \pr{color} argument to
  fill the background of the drawing section (it will always be
  white).}
\\[\dt]
\pr{void} \pr{F.newpage()}\\
\tab
\parbox[t]{\wi}{same as \pr{F.clear()}}
\\[\dt]
\pr{void} \pr{F.newfile(double w, double h=HEIGHT, string
  name="unnamed.ps")}\\
\tab
\parbox[t]{\wi}{associates a new \ps\ file with the current \psf\
  object. If necessary (i.e., if the user did not call \pr{F.close()}
  before), the old file is closed and written to the current working
  directory. Essentially, this function behaves as if a new \psf\
  object was created using a constructor. In particular, all parameter
  values are reset to their default values (cf. 2.3.2).}
\\[\dt]
\pr{void} \pr{F.newfile(string name="unnamed.ps")}\\
\tab
\parbox[t]{\wi}{same as \pr{F.newfile(WIDTH,HEIGHT,name)}}
\\[\dt]
\pr{void} \pr{F.close()}\\
\tab
\parbox[t]{\wi}{closes the file after adding some trailing information
  to it (number of output pages etc.). Calling this function is
  optional (if necessary, it is called automatically at the end of the
  lifetime of the variable~\pr{F}).}
%
\subsubsection{Setting parameters}
%
\tp{bool}\pr{set\_draw\_bb(bool flag)}\\
\tab
\parbox[t]{\wi}{sets the \emph{draw bounding box} parameter to
  \pr{flag}. Possible values are \pr{true} and \pr{false}; default is
  \pr{true}.

  This method can be called anywhere \emph{before} the page is
  \emph{finished} that the method first applies to, i.e. before the
  next \pr{clear} (or \pr{newpage}) or \pr{close}.}
\\[\dt]
\tp{output\_mode}\pr{set\_output\_mode(output\_mode new\_mode)}\\
\tab
\parbox[t]{\wi}{sets the \emph{ouptput mode} parameter to
  \pr{new\_mode}. Possible values are \pr{colored\_mode} and \pr{gray\_mode};
  default is \pr{colored\_mode}.}
\\[\dt]
\tp{double}\pr{set\_gray\_value(color c, double d)}\\
\tab
\parbox[t]{\wi}{sets the gray value of the color \pr{c} to
  \pr{d}. Possible values are: any color for \pr{c}, and a number in
  the range from 0 (black) through 1 (white) for \pr{d}. Default
  values are adopted from the \ps\ language. (See also Colors in
  section 1.)}
\\[\dt]
\tp{color}\pr{set\_color(color c)}\\
\tab
\parbox[t]{\wi}{sets the \emph{foreground color} parameter to
  \pr{c}. \pr{c} may be any color: either one of the 16 predefined
  ones or a user defined color. Default is \pr{black}.}
%\\[\dt]
% Style manipulation!

\pagebreak
\noindent
\tp{double}\pr{set\_line\_width(double pix)}\\
\tab
\parbox[t]{\wi}{sets the \emph{linewidth} parameter to \pr{pix}
  units. Default is 1. The actual linewidth later visible in the
  output depends on the \emph{unitlinewidth} parameter (see next).}
\\[\dt]
\tp{double}\pr{set\_unit\_line\_width(double cm)}\\
\tab
\parbox[t]{\wi}{sets the \emph{unitlinewidth} parameter to \pr{cm}
  cm. Default is the value of the \pr{PIXEL} symbolic constant
  (2.54cm/72), which is the width of a typographical dot. This
  constant may be used to redefine this parameter value.

  Arguments less than the value of PIXEL (like 0.01 [cm] or
  $\mbox{\pr{PIXEL}}*0.5$) are also permitted.}
\\[\dt]
\tp{line\_style}\pr{set\_line\_style(line\_style s)}\\
\tab
\parbox[t]{\wi}{sets the \emph{linestyle} parameter to \pr{s}. Possible
  values are \pr{solid}, \pr{dashed}, and \pr{dotted}. Defaults is
  \pr{solid}.}
\\[\dt]
\tp{double}\pr{set\_node\_width(double pix)}\\
\tab
\parbox[t]{\wi}{sets the \emph{nodewidth} parameter to \pr{pix}
  pixels. Default value is 20.}
\\[\dt]
\tp{double}\pr{set\_cross\_width(double cm)}\\
\tab
\parbox[t]{\wi}{sets the \emph{crosswidth} parameter to \pr{cm} cm.
  Default is 0.2.}
\\[\dt]
\tp{double}\pr{set\_arrow\_length(double cm)}\\
\tab
\parbox[t]{\wi}{sets the \emph{arrowlength} parameter to \pr{cm} cm.
  Default is 0.5.}
\\[\dt]
\tp{double}\pr{set\_arrow\_angle(double rad)}\\
\tab
\parbox[t]{\wi}{sets the \emph{arrowangle} parameter to \pr{rad} rad.
  Default is $\pi/10$. The \pr{PI} symbolic constant can be used to
  reset the arrow angle.}
\\[\dt]
\tp{string}\pr{set\_text\_font(string s)}\\
\tab
\parbox[t]{\wi}{sets the \emph{textfont} parameter to \pr{s}. Possible
  values are names of fonts that are interpretable by
  \ps\footnotemark. Default is \pr{"Helvetica"}, which is close to the
  font used in windows.}
\footnotetext{Such are: \pr{"Times-Roman"}, \pr{"Times-Italic"},
\pr{"Times-Bold"}, \pr{"Times-BoldItalic"}, \pr{"Helveti\-ca"},
\pr{"Helvetica-Oblique"}, \pr{"Helvetica-Bold"},
\pr{"Helvetica-BoldOblique"}, \pr{"Courier"},
\pr{"Courier\--Oblique"}, \pr{"Courier-Bold"},
\pr{"Courier-BoldOblique"}, and \pr{"Symbol"}}
\\[\dt]
\tp{double}\pr{set\_font\_size(double cm)}\\
\tab
\parbox[t]{\wi}{sets the \emph{fontsize} parameter to \pr{cm} cm.
  Default is 0.4. This value approximately corresponds to the
  \emph{height} of the letter \pr{M} in the current font. Note that
  fonts also become \emph{wider} when the fontsize is increased.}
\\[\dt]
All setting parameter operations return the previous value of the
parameter.
%
\subsubsection{Reading parameters}
%
\tp{bool}\pr{get\_draw\_bb()}\\
\tp{output\_mode}\pr{get\_output\_mode()}\\
\tp{double}\pr{get\_gray\_value(color c)}\\
\tab
\parbox[t]{\wi}{gets the gray value that has been assigned to color
  \pr{c} by the user. If none is assigned, -1 is returned.}\\
\tp{color}\pr{get\_color()}\\
\tp{double}\pr{get\_line\_width()}\\
\tp{double}\pr{get\_unit\_line\_width()}\\
\tp{line\_style}\pr{get\_line\_style()}\\
\tp{double}\pr{get\_node\_width()}\\
\tp{double}\pr{get\_cross\_width()}\\
\tp{double}\pr{get\_arrow\_length()}\\
\tp{double}\pr{get\_arrow\_angle()}\\
\tp{string}\pr{get\_text\_font()}\\
\tp{double}\pr{get\_font\_size()}\\[\dt]
These functions return the value of the corresponding \psf\ parameter.
%
\subsubsection{Drawing Operations}
%
All drawing operations have an optional color argument at the end of
the parameter list. If this argument is omitted, the current
foreground color is used instead.

\vspace{\dt}
In the following, only a list of all available operations is given,
but without explanations of what they draw. This can be obtained from
the \l\ manual.
%
\subsubsection*{2.3.4.1 Drawing points}
%
\pr{void} \pr{F.draw\_point(double x, double y, color c=fgcol)}\\
\pr{void} \pr{F.draw\_point(point p, color c=fgcol)}\\
\pr{void} \pr{F.draw\_pixel(double x, double y, color c=fgcol)}\\
\pr{void} \pr{F.draw pixel(point p, color c=fgcol)}
%
\subsubsection*{2.3.4.2 Drawing line segments}
%
\pr{void} \pr{F.draw\_segment(double x1, double y1, double x2, double y2,}\\
  \tab\pr{color c=fgcol)}\\
\pr{void} \pr{F.draw\_segment(point p, point q, color c=fgcol)}\\
\pr{void} \pr{F.draw\_segment(segment s, color c=fgcol)}
%
\subsubsection*{2.3.4.3 Drawing lines}
%
\pr{void} \pr{F.draw\_line(double x1, double y1, double x2, double  y2,}\\
  \tab\pr{color c=fgcol)}\\
\pr{void} \pr{F.draw\_line(point p, point q, color c=fgcol)}\\
\pr{void} \pr{F.draw\_line(segment s, color c=fgcol)}\\
\pr{void} \pr{F.draw\_line(line l, color c=fgcol)}\\
\pr{void} \pr{F.draw\_hline(double y, color c=fgcol)}\\
\pr{void} \pr{F.draw\_vline(double x, color c=fgcol)}
%
\subsubsection*{2.3.4.4 Drawing arrows}
%
\pr{void} \pr{F.draw\_arrow\_head(point p, double dir, color c=fgcol)}\\
\pr{void} \pr{F.draw\_arrow(double x1, double y1, double x2, double y2,}\\
  \tab\pr{color c=fgcol)}\\
\pr{void} \pr{F.draw\_arrow(point p, point q, color c=fgcol)}\\
\pr{void} \pr{F.draw\_arrow(segment s, color c=fgcol)}
%
\subsubsection*{2.3.4.5 Drawing circles}
%
\pr{void} \pr{F.draw\_circle(double x, double y, double r, color c=fgcol)}\\
\pr{void} \pr{F.draw\_circle(point p, double r, color c=fgcol)}\\
\pr{void} \pr{F.draw\_circle(circle C, color c=fgcol)}\\
\pr{void} \pr{F.draw\_ellipse(double x, double y, double r1, double r2,}\\
  \tab\pr{color c=fgcol)}\\
\pr{void} \pr{F.draw\_ellipse(point p, double r1, double r2, color c=fgcol)}
%
\subsubsection*{2.3.4.6 Drawing discs}
%
\pr{void} \pr{F.draw\_disc(double x, double y, double r, color c=fgcol)}\\
\pr{void} \pr{F.draw\_disc(point p, double r, color c=fgcol)}\\
\pr{void} \pr{F.draw\_disc(circle C, color c=fgcol)}\\
\pr{void} \pr{F.draw\_filled\_ellipse(double x, double y, double r1, double r2,}\\
  \tab\pr{color c=fgcol)}\\
\pr{void} \pr{F.draw\_filled\_ellipse(point p, double r1, double r2,}\\
  \tab\pr{color c=fgcol)}
%
\subsubsection*{2.3.4.7 Drawing polygons}
%
\pr{void} \pr{F.draw\_polygon(list<point> lp, color c=fgcol)}\\
\pr{void} \pr{F.draw\_polygon(polygon P, color c=fgcol)}\\
\pr{void} \pr{F.draw\_filled\_polygon(list<point> lp, color c=fgcol)}\\
\pr{void} \pr{F.draw\_filled\_polygon(polygon P, color c=fgcol)}\\
\pr{void} \pr{F.draw\_rectangle(double x1, double y1, double x2, double y2,}\\
  \tab\pr{color c=fgcol)}\\
\pr{void} \pr{F.draw\_rectangle(point p, point q, color c=fgcol)}\\
\pr{void} \pr{F.draw\_box(double x1, double y1, double x2, double y2,}\\
  \tab\pr{color c=fgcol)}\\
\pr{void} \pr{F.draw\_box(point p, point q, color c=fgcol)}\\
\pr{void} \pr{F.draw\_triangle(point a, point b, point c, color c=fgcol)}\\
\pr{void} \pr{F.draw\_filled\_triangle(point a, point b, point c,}\\
  \tab\pr{color c=fgcol)}
%
\subsubsection*{2.3.4.8 Drawing text}
%
\pr{void} \pr{F.draw\_text(double x, double y, string s, color c=fgcol)}\\
\pr{void} \pr{F.draw\_text(point p, string s, color c=fgcol)}\\
\pr{void} \pr{F.draw\_ctext(double x, double y, string s, color c=fgcol)}\\
\pr{void} \pr{F.draw\_ctext(point p, string s, color c=fgcol)}
%
\subsubsection*{2.3.4.9 Drawing nodes}
%
\pr{void} \pr{F.draw\_node(double x, double y, color c=fgcol)}\\
\pr{void} \pr{F.draw\_node(point p, color c=fgcol)}\\
\pr{void} \pr{F.draw\_filled\_node(double x, double y, color c=fgcol)}\\
\pr{void} \pr{F.draw\_filled\_node(point p, color c=fgcol)}\\
\pr{void} \pr{F.draw\_text\_node(double x, double y, string s, color c=fgcol)}\\
\pr{void} \pr{F.draw\_text\_node(point p, string s, color c=fgcol)}\\
\pr{void} \pr{F.draw\_int\_node(double x, double y, int i, color c=fgcol)}\\
\pr{void} \pr{F.draw\_int\_node(point p, int i, color c=fgcol)}
%
\subsubsection*{2.3.4.10 Drawing edges}
%
\pr{void} \pr{F.draw\_edge(double x1, double y1, double x2, double y2,}\\
  \tab\pr{color c=fgcol)}\\
\pr{void} \pr{F.draw\_edge(point p, point q, color c=fgcol)}\\
\pr{void} \pr{F.draw\_edge(segment s, color c=fgcol)}\\
\pr{void} \pr{F.draw\_edge\_arrow(double x1, double y1, double x2, double y2,}\\
  \tab\pr{color c=fgcol)}\\
\pr{void} \pr{F.draw\_edge\_arrow(point p, point q, color c=fgcol)}\\
\pr{void} \pr{F.draw\_edge\_arrow(segment s, color c=fgcol)}

\vspace{\dt}
\emph{Not} (yet) implemented have been drawing operations for objects
involving an \emph{arc} (like \pr{draw\_arc}, \pr{draw\_arc\_arrow},
\pr{draw\_arc\_edge}), \emph{functions} (like \pr{plot\_xy}), and
those involving \emph{messages}. Whenever it is attempted to call a
function that is not implemented, a warning is issued both on standard
error output (screen) and in the \ps\ file that the output is written into.
%
\subsection{Output operators}
%
For output of basic geometric objects, namely points, lines, segments,
circles, and polygons, the \pr{<<} - operator can be used in the
following way:

\vspace{\dt}
\noindent
\makebox[5cm][l]{\pr{F << point p}}   like \pr{F.draw\_point(p)}\\
\makebox[5cm][l]{\pr{F << segment s}} like \pr{F.draw\_segment(s)}\\
\makebox[5cm][l]{\pr{F << line l}}    like \pr{F.draw\_line(l)}\\
\makebox[5cm][l]{\pr{F << circle C}}  like \pr{F.draw\_circle(C)}\\
\makebox[5cm][l]{\pr{F << polygon P}} like \pr{F.draw\_polygon(P)}

\vspace{\dt}
\noindent
Note that there is no way to pass a color different from the current
foreground color to a drawing operation employing the \pr{<<} -
operator.
%
\end{document}
