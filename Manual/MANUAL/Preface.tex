\chapter{Preface} \label{Preface}

One of the major differences between combinatorial computing and other areas of
computing such as statistics, numerical analysis and linear programming is the
use of complex data types.  Whilst the built-in types, such as integers, reals,
vectors, and matrices, usually suffice in the other areas, combinatorial
computing relies heavily on types like stacks, queues, dictionaries, sequences,
sorted sequences, priority queues, graphs, points, segments, $\ldots$ In the
fall of 1988, we started a project (called {\bf LEDA} for Library of Efficient
Data types and Algorithms) to build a small, but growing library of data types
and algorithms in a form which allows them to be used by non-experts. We hope
that the system will narrow the gap between algorithms research, teaching, and
implementation. The main features of LEDA are:

\begin{enumerate}
\item 
    LEDA provides a sizable collection of data types and algorithms in a form 
    which allows them to be used by non-experts. This
    collection includes most of the data types and algorithms described in the 
    text books of the area. 

\item 
    LEDA gives a precise and readable specification for each of the data types 
    and algorithms mentioned above.  The specifications are short (typically, 
    not more than a page), general (so as to allow several implementations), 
    and abstract (so as to hide all details of the implementation). 

\item
    For many efficient data structures access by position is important. In 
    LEDA, we use an item concept to cast positions into an abstract form. We 
    mention that most of the specifications given in the LEDA manual use this 
    concept, i.e., the concept is adequate for the description of many data 
    types. 

\item
    LEDA contains efficient implementations for each of the data types, e.g., 
    Fibonacci heaps for priority queues, skip lists and dynamic perfect 
    hashing for dictionaries, ...


\item
    LEDA contains a comfortable data type graph. It offers the standard 
    iterations such as ``for all nodes v of a graph G do'' or ``for all 
    neighbors w of v do'', it allows to add and delete vertices and edges 
    and it offers arrays and matrices indexed by nodes and edges,...  
    The data type graph allows to write programs for graph problems in a 
    form close to the typical text book presentation.


\item 
    LEDA is implemented by a \CC{} class library. It can be used with almost
    any \CC{} compiler that supports templates. 


\item
    LEDA is available from Algorithmic Solutions Software GmbH. See 
    \htmladdnormallink{\texttt{http://www.algorithmic-solutions.com}}{http://www.algorithmic-solutions.com}. 

\end{enumerate}

This manual contains the specifications of all data types and algorithms 
currently available in LEDA. Users should be familiar with the \CC{}
programming language (see \cite{S91} or \cite{Li89}).  

The manual is structured as follows: In Chapter 
\htmlref{Basics}{Basics}, which is a
prerequisite for all other chapters, we discuss the basic concepts and
notations used in LEDA. New users of LEDA should carefully read Section
\htmlref{User Defined Parameter Types}{User Defined Parameter Types} to avoid problems when plugging in self
defined parameter types. If you want to get information about the LEDA
documentation scheme please read Section \htmlref{DocTools}{DocTools}. 
For technical
information concerning the installation and usage of LEDA 
users should refer to Chapter
\htmlref{TechnicalInformation}{TechnicalInformation}. There is also a section describing 
namespaces and the interaction with other software
libraries (Section \htmlref{NameSpace}{NameSpace}).  The other chapters define the data types and algorithms available
in LEDA and give examples of their use. These chapters can be consulted
independently from one another.

More information about LEDA can be found on the LEDA web page:\\
\htmladdnormallink{\texttt{http://www.algorithmic-solutions.com/leda/}}{http://www.algorithmic-solutions.com/leda}

Finally there's a tool called \verb|xlman| which allows online help and
demonstration on all unix platforms having a \LaTeX{} package installed.

\bigskip
{\large\bf New in Version \ledaversion}

Please read the CHANGES and FIXES files in the LEDA root directory for more 
information.
