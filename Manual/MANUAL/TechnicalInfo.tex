\chapter{Technical Information} 
\label{TechnicalInformation}

This chapter provides information about installation and usage of LEDA, 
the interaction with other software packages, and an overview
of all currently supported system platforms. 

\section{LEDA Library and Packages} 
\label{Libraries}

The implementations of most LEDA data types and algorithms are precompiled and
contained in one library libleda that can be linked with \CC\ application programs. 

LEDA is available either as source code package or as object code package for 
the platforms listed in Section \htmlref{Platforms}{Platforms}. 
Information on how to obtain LEDA can be found at
\htmlref{http://www.algorithmic-solutions.com/index.php/products/leda-for-c}{http://www.algorithmic-solutions.com/index.php/products/leda-for-c}

Sections \htmlref{Source Contents}{Source Contents} ff. describe how to compile the LEDA libraries
in the source code package for Unix (including Linux and CygWin) 
and Microsoft Windows. Section 
\htmlref{http://www.algorithmic-solutions.info/leda\_manual/Object\_Code\_on.html}{UnixObjectCodePackage} and Section 
\htmlref{http://www.algorithmic-solutions.info/leda\_manual/DLL\_s\_MS\_Visual.html}{WinObjectCodePackage} 
describe the installation and usage of the object code packages for Unix
and Windows, respectively.

\section{Contents of a LEDA Source Code Package}
\label{Source Contents}

The main directory of the GUI source code package should contain at 
least the following files and subdirectories:

\begin{center}
\begin{tabular}{ll}
Readme.txt               & Readme File\\
CHANGES  (please read !) & most recent changes\\
FIXES                    & bug fixes since last release\\
license.txt              & license text\\
lconfig                  & configuration command for unix\\
lconfig.bat              & configuration command for windows\\
Makefile                 & make script\\
confdir/                 & configuration directory\\
incl/                    & include directory\\
src/                     & source files compiled into the LEDA Free Edition\\
src1/					 & other source files\\
test/                    & example and test programs\\
demo/                    & demo programs\\
\end{tabular}
\end{center}

\section{Source Code on UNIX Platforms}
\subsection*{Source Code Configuration on UNIX}

Important remark: When compiling the sources on Unix- or Linuxsystems the development packages
for X11 and Xft should be installed. On Ubuntu, for instance, you should call\\
\ \\
sudo apt-get install libx11-dev\\
sudo apt-get install libxft-dev\\
\ \\
\begin{enumerate}

\item Go to the LEDA main directory.

\item Type: \texttt{lconfig <cc>  [static | shared]}\\
\ \\
where \texttt{<cc>} is the name (or command) of your \CC\ compiler
and the optional second parameter defines the kind of libraries
to be generated. Please note that as far as Unix systems go, we currently
only support several Linux distributions. LEDA might work on other Unix systems, too - it
was originally developed, for instance, on SunOS - but there is no guarantee for that.\\
\ \\
Examples: \texttt{lconfig CC}, \texttt{lconfig g++}, \texttt{lconfig 
sunpro shared}\\
\ \\
\texttt{lconfig} without arguments prints a list of known compilers.\\
If your compiler is not in the list you might have to
edit the \texttt{<LEDA/sys/unix.h>} header file.
\end{enumerate}

\subsection*{LEDA Compilation on UNIX}

Type \texttt{make} for building the object code library
libleda.a   (libleda.so if shared libraries are used). The make command will
also have another library created named libGeoW.a; it only deals with the data type GeoWin.
There is no shared version of the this library available.

Now follow the instructions given in  Section \htmlref{UnixObjectCodePackage}{UnixObjectCodePackage}.

\section{Source Code on Windows with MS Visual \CC}
\subsection*{Source Code Configuration for MS Visual \CC}
\begin{enumerate}
\item Setting the Environment Variables for Visual \CC:\\
The compiler \texttt{CL.EXE} and the linker \texttt{LINK.EXE} require that 
the environment variables \texttt{PATH, INCLUDE}, and \texttt{LIB} have been 
set properly.\\
Therefore, when compiling LEDA, simply open the proper command prompt that comes
with the Visual Studio. The environment variables are then set as required. Just 
start the x86 (when compiling for a 32 bit platform) or
the x64 (when compiling for a 64 bit platform) Native Tools Command Prompt.
\item Go to the LEDA main directory.
\item Type: \texttt{lconfig [msc | msc-mt | msc-mt | msc64-mt | msc-mt-15 | msc64-mt-15 ] [dll] [ md | mt | mdd | mtd ]}
\end{enumerate}
        
{\bf Remark:} When using MS Visual \CC to compile LEDA you have to choose msc for
32 bit single-threaded compilation, msc-mt for 32 bit multi-threaded compilation, msc64
for 64 bit single-threaded compilation, and msc64-mt for 64 bit multi-threaded compilation.
When using MS Visual Studio 2015 or later Visual Studio versions, you should use msc-mt-15
and msc64-mt-15 respectively.
When building an application with LEDA and MS Visual Studio \CC 
the LEDA library you use depends on the Microsoft C 
runtime library you intend to link with. Your application code and LEDA both must be
linked to the same Microsoft C runtime library; otherwise serious linker or runtime 
errors may occur. The Microsoft C runtime libraries are related to the compiler 
options as follows

\begin{center}
\begin{tabular}{|l|l|}
\hline
           C Runtime Library         &   Option\\ \hline

           LIBCMT.LIB                 &  -MT\\
           LIBCMTD.LIB                &  -MTd\\
           MSVCRT.LIB                 &  -MD\\
           MSVCRTD.LIB                &  -MDd\\
\hline
\end{tabular}
\end{center}

           In order to get the suitable Libs or DLL please choose the 
           corresponding option in the call of \texttt{lconfig}.

\subsection*{LEDA Compilation with MS Visual \CC}

       Type \texttt{make\_lib} for building the object code libraries 
\begin{center}
\begin{tabular}{lll}
       static: & libleda.lib&   LEDA library without GeoWin\\
               & libGeoW.lib&   GeoWin library\\
&&\\
      dynamic: & leda.dll, leda.lib&\\
&libgeow.lib&
\end{tabular}
\end{center}
{\bf Remarks:} The current LEDA package supports only the dynamic version; therefore setting dll
in the lconfig call is mandatory at the moment. GeoWin is currently not available as a DLL
and will always be build as a static library.

\medskip
Now follow the instructions given in the corresponding section for the 
Windows object code package (Section \htmlref{WinObjectCodePackage}{WinObjectCodePackage} ff.).


\section{Usage of Header Files}\label{Header Files}
LEDA data types and algorithms can be used in any \CC\ program as described 
in this manual (for the general layout of a manual page please see Chapter
\htmlref{LEDA Manual Page}{LEDA Manual Page}). The specifications (class declarations) are contained
in header files. To use a specific data type its header file has to be 
included into the program. In general the header file for data type xyz is 
\texttt{<LEDA/group/xyz.h>}. The correct choice for group and xyz is specified on the 
type's manual page.

\section{Object Code on UNIX}
\label{UnixObjectCodePackage}


\subsection*{Files and Directories}
To compile and link your programs with LEDA, the LEDA main directory should
contain at least the following files and subdirectories:
\begin{center}
\begin{tabular}{ll}
Readme.txt               & Readme File\\
Install/unix.txt & txt--version of this section\\
incl/ & the LEDA include directory\\
libleda.a (libleda.so) & the LEDA library
\end{tabular}
\end{center}

  The static library has the extension \texttt{.a}.  
  If a shared library is provided it has extension \texttt{.so}.

\subsection*{Preparations}
Unpacking the LEDA distribution file 
\texttt{LEDA-<ver>-<sys>-<cc>.tar.gz}
will create the LEDA root directory 
"\texttt{LEDA-<ver>-<sys>-<cc>}". You
might want to rename it or move it to some different place. Let 
\texttt{<LEDA>} denote
the final complete path name of the LEDA root directory.

To install and use the Unix object code of LEDA you have to modify your 
environment as follows:
\begin{itemize}
   \item
       Set the environment variable \texttt{LEDAROOT} to the LEDA root 
       directory:

\begin{center}
\begin{tabular}{ll}
       csh/tcsh:  & \texttt{setenv LEDAROOT <LEDA>}\\
&\\
       sh/bash: &   \texttt{LEDAROOT=<LEDA>}\\
                 &  \texttt{export LEDAROOT}
\end{tabular}
\end{center}

  \item  Shared Library:   (for solaris, linux, irix, osf1)\\
       If you planning to use the shared library include 
       \texttt{\$LEDAROOT} into 
       the \texttt{LD\_LIBRARY\_PATH} search path.

  \item Make sure that the development packages for
		X11 and Xft have been installed. On Ubuntu, for instance, you should have called\\
		sudo apt-get install libx11-dev\\
		sudo apt-get install libxft-dev\\
\end{itemize}

\subsection*{Compiling and Linking Application Programs}
\begin{enumerate}
\item  Use the -I compiler flag to tell the compiler where to find the LEDA 
      header files.
\begin{verbatim}
      CC (g++) -I$LEDAROOT/incl -c file.cpp
\end{verbatim}

\item Use the -L compiler flag to tell the compiler where to find the 
      library.
\begin{verbatim}
      CC (g++)  -L$LEDAROOT file.o -lleda -lX11 -lXft -lm
\end{verbatim}

      When using graphics on Solaris systems you might have to link with the 
      system socket library and the network services library as well:
\begin{verbatim}
      CC (g++)  ... -lleda -lX11 -lXft -lsocket -lnsl -lm
\end{verbatim}
{\bf Remark:} The libraries must be given in the above order. 


\item Compile and link simultaneously with
\begin{verbatim}
      CC (g++)  -I$LEDAROOT/incl -L$LEDAROOT file.c  -lleda -lX11 -lXft -lm
\end{verbatim}
\end{enumerate}

   When using the multi-threaded version of LEDA you also have to set the flags 
   \texttt{LEDA\_MULTI\_THREAD} and pthread during compilation 
   (\texttt{-DLEDA\_MULTI\_THREAD} -pthread) and you have to additionally link against
   the pthread library (-pthread).
   You may want to ask your system administrator to install the header files 
   and libraries in the system's default directories.
   Then you no longer have to specify header and library search paths on the 
   compiler command line.


\subsection*{Example programs and demos}
   The source code of all example and demo programs can be found in 
   \$LEDAROOT/test and \$LEDAROOT/demo. Goto \$LEDAROOT/test or 
   \$LEDAROOT/demo and type \texttt{make} to compile and link all test or demo 
   programs, respectively.

            
\noindent
{\bf Important Remark:} When using g++ version 4.x.x with optimization level 2 (-O2) or
higher, you should always compile your sources setting the following flag:
\smallskip\noindent
-fno-strict-aliasing
                                                    

\section{Static Libraries for MS Visual \CC\ .NET} 
This section describes the installation and usage of static libraries of
LEDA with Microsoft Visual \CC\ .NET.

\medskip
{\bf Remark:} The current LEDA package is delivered with dynamic libraries. So this section
is only relevant to you if you created static libraries from the source code.

\subsection*{Preparations}
   To install LEDA you only need to execute the LEDA distribution
   file \texttt{LEDA-<ver>-<package>-win32-<compiler>.exe}.
   During setup you can choose the name of the LEDA root directory and the 
parts of LEDA you want to install.
\

   Then you have to set the environment variable \texttt{LEDAROOT}. On MS Windows 10
   this can be done as follows:

\
\begin{description}
\item[MS Windows 10:]\
\begin{enumerate}
\item Open the Start Search, type in �env�, and choose �Edit the system 
environment variables�. A window titled "System Properties" should open.
\item  Click the button "Environment variables..." in the lower right corner of the
"System Properties" window. A new window opens that allows to add/change/delete the user
variables for your account as well as the system variables, provided you have admin
rights. If not, change the environment variables of your account. 

             Add a new user variable \texttt{LEDAROOT} with value \texttt{<LEDA>}.
\end{enumerate}
\end{description}

In case you are working on a different version of MS Windows, please consult the
documentation of your version in order to learn how to perform the corresponding steps.
You might have to restart your computer for the changes to take effect.

\subsection*{Files and Directories}
To compile and link your programs with LEDA, the LEDA main directory should
contain the following files and subdirectories:
\begin{center}
\begin{tabular}{ll}
Readme.txt               & Readme File\\
incl$\backslash$ & the LEDA include directory\\
\end{tabular}
\end{center}
and at least one of the following library sets
\begin{itemize}
\item \texttt{
   libleda\_md.lib,
   libgeow\_md.lib}
\item \texttt{
   libleda\_mdd.lib,
   libgeow\_mdd.lib}
\item \texttt{
   libleda\_mt.lib,
   libgeow\_mt.lib}
\item \texttt{
   libleda\_mtd.lib,
   libgeow\_mtd.lib}
\end{itemize}

\subsection*{Compiling and Linking in Microsoft Visual \CC\ .NET}
	We now explain how to proceed in order to compile and link an application program
	using LEDA with MS Visual Studio 2017. If you are using a different version of
	MS Visual Studio, please read and understand the guidelines below and consult the
	documentation of your version of the Studio in order to learn how to perform the
	corresponding steps.

\begin{description}
\item[(1)] In the "File" menu of Visual \CC\ .NET click on "New-->Project".
\item[(2)] Choose "Visual C++" as \texttt{project type} and choose "Empty Project".
\item[(3)] Enter a project name, choose a directory for the project, and click "OK".
\item[(4)] After clicking "OK" you have an empty project space. Choose, for instance,
"Debug" and "x64" (or "x86" in case you are working on a 32-bit system) in the
corresponding pick lists.
\\
\item[If you already have a source file \texttt{prog.cpp}:]
\item[(5)] Activate the file browser and add \texttt{prog.cpp} to the main
 folder  of your project
\item[(6)] In the \texttt{Solution Explorer} of your project click on 
"Source Files" with the right mouse button, then click on 
"Add--> Add Existing Item" with the left mouse button
\item[(7)] Double click on \texttt{prog.cpp}
\\
\item[If you want to enter a new source file:] 
\item[(5')] In the \texttt{Solution Explorer} of your project click on 
"Source Files" with the right mouse button, then click on 
"Add--> Add New Item" with the left mouse button.
\item[(6')] Choose "C++ File" in \texttt{Templates}, enter a name, 
and click "Add".
\item[(7')] Enter your code.
\\
\item[(8)] In the \texttt{Solution Explorer} right click on your project
and left click on "Properties"
\item[(9)] Click on  "C/\CC" and "Code Generation" and choose 
the "Run Time Library" (=compiler flag) you want to use.

 If you chose "Debug" in step 4, the default value is now "/MDd", alternatives are
      "/MD", "/MT", and "/MTd". 
      Notice that you have to use the LEDA libraries that correspond to the 
      chosen flag, e.g., with option "/MDd" you must use 
      \texttt{libleda\_mdd.lib} and \texttt{libgeow\_mdd.lib}.
      Using another set of libraries with "/MDd" could lead to serious 
      linker errors.
\item[(10)] Click on "Linker" and "Command Line" and add the name of the 
LEDA libraries you want to
       use in "Additional Options" as follows.
       We use \texttt{<opt>} to indicate the compiler option chosen in 
Step (9) 
       (e.g., \texttt{<opt>} is \texttt{mdd} for "/MDd").
\begin{itemize}   
          \item \texttt{libleda\_$<$opt$>$.lib}  \\
            for programs using data types of LEDA but not GeoWin.
    
          \item \texttt{libgeow\_$<$opt$>$.lib libleda\_$<$opt$>$.lib}\\
            for programs using GeoWin
\end{itemize}                

\item[(11)] Click on "V\CC\ Directories" of the "Properties" window.
\item[(12)] Choose "Include Files" and add the 
       directory \texttt{$<$LEDA$>$$\backslash$incl} containing the LEDA 
include files 
       (Click on the line starting with "Include Files", then click on "Edit..." in
	   the pick list at the right end of that line. Push the "New line" button and then enter 
       \texttt{$<$LEDA$>$$\backslash$incl}, or click on the small grey 
       rectangle on the right and
       choose the correct directory.) Alternatively you can click C/\CC--> General in the 
	   Configuration Properties and then edit the line "Additional Include Directories".
\item[(13)] Choose "Library Directories" and add the 
       directory \texttt{$<$LEDA$>$} containing the LEDA libraries.
\item[(14)] Click "OK" to leave the "Properties".
\item[(15)] In the "Build" menu click on \texttt{"<Build Project>"} or 
\texttt{"Rebuild <Project>"} to compile your program.
\item[(16)] In order to execute your program, click the green play button in the tool bar.
\end{description}

   Remark: If your \CC\ source code files has extension .c, you need to add 
            the option "/TP" in "Project Options" (similar to Step (9)), 
            otherwise you will get a number of compiler errors. 
(Click on  "C/\CC" and "Command Line". Add \texttt{/TP} in 
"Additional Options" and click "Apply".)

To add LEDA to an existing Project in Microsoft Visual \CC\ .NET, start the 
Microsoft Visual Studio with your project and follow Steps (8)--(14) above.

\subsection*{Compiling and Linking Application Programs in a DOS-Box}
\ \\
\begin{description}
\item[(a) Setting the Environment Variables for Visual \CC:]
\ \\
The compiler \texttt{CL.EXE} and the linker \texttt{LINK.EXE} require 
that the environment variables \texttt{PATH, INCLUDE}, and \texttt{LIB} have 
been set properly. This can easily be ensured by using the command prompts that are
installed on your computer with your Visual Studio installation.\\
          
      To compile programs together with LEDA, the environment 
      variables \texttt{PATH}, \texttt{LIB}, and \texttt{INCLUDE} must additionally contain the 
      corresponding LEDA directories. We now explain how to do that with MS Windows 10.
	  If you are using a different version of MS Windows, please read and understand the
	  guidelines below and consult the documentation of your operating system in order to
	  learn how to perform the corresponding steps.
 
\item[(b) Setting Environment Variables for LEDA:]
\
\begin{description}
\item[MS Windows 10:]\
\begin{enumerate}
\item Open the Start Search, type in �env�, and choose �Edit the system 
environment variables�. A window titled "System Properties" should open.\\
\ \\
\item  Click the button "Environment variables..." in the lower right corner of the
"System Properties" window. A new window opens that allows to add/change/delete the user
variables for your account as well as the system variables, provided you have admin
rights. If not, change the environment variables of your account. 

             If a user variable \texttt{PATH}, \texttt{LIB}, or
\texttt{INCLUDE} already exists, extend the current value as follows:

\begin{itemize}
             \item extend \texttt{PATH} by \texttt{<LEDA>}
             \item extend \texttt{INCLUDE} by \texttt{<LEDA>$\backslash$incl}
             \item extend \texttt{LIB} by \texttt{<LEDA>}
\end{itemize}

             Otherwise add a new user variable \texttt{PATH, INCLUDE}, or
\texttt{LIB} with value \texttt{<LEDA>}, respectively 
\texttt{<LEDA>$\backslash$incl}.
\end{enumerate}
\end{description}

      You might have to restart your computer for the changes to take effect.

\item[(c) Compiling and Linking Application Programs:]
\ \\
    
	  After setting the environment variables, you can use the LEDA libraries 
as 
      follows to compile and link programs.

      Programs that do not use GeoWin:
\begin{verbatim}
cl <option> prog.cpp libleda.lib
\end{verbatim}

      Programs using GeoWin:
\begin{verbatim}
cl <option> prog.cpp libGeoW.lib libleda.lib 
\end{verbatim} 

Possible values for \texttt{<option>} are "-MD", 
"-MDd", "-MT", and "-MTd". You have to use the LEDA libraries that 
correspond to the chosen \texttt{<option>}, e.g., with option "-MD" you 
must use \texttt{libleda\_md.lib}.
      Using another set of libraries with "-MD" could lead to serious 
      linker errors.


\end{description}

\subsection*{Example programs and demos}
   The source code of all example and demo programs can be found in 
   the directory \texttt{<LEDA>$\backslash$test} and 
   \texttt{<LEDA>$\backslash$demo}. 
   Goto \texttt{<LEDA>} and type
   \texttt{make\_test} or \texttt{make\_demo} to compile and link all test or 
   demo programs, respectively. 

\section{DLL's for MS Visual \CC\ .NET} 
This section describes the installation and usage of
LEDA Dynamic Link Libraries (DLL's) with Microsoft Visual \CC\ .NET.


\subsection*{Preparations}

   To install LEDA you only need to execute the LEDA distribution
   file \texttt{LEDA-<ver>-<package>-win32-<compiler>.exe}.
   During setup you can  choose the name of the LEDA root directory and the 
parts of LEDA you want to install.
 

   Then you have to set the environment variable \texttt{LEDAROOT}. On MS Windows 10
   this can be done as follows:

\
\begin{description}
\item[MS Windows 10:]\
\begin{enumerate}
\item Open the Start Search, type in �env�, and choose �Edit the system 
environment variables�. A window titled "System Properties" should open.\\
\ \\
\item  Click the button "Environment variables..." in the lower right corner of the
"System Properties" window. A new window opens that allows to add/change/delete the user
variables for your account as well as the system variables, provided you have admin
rights. If not, change the environment variables of your account. 

             Add a new user variable \texttt{LEDAROOT} with value \texttt{<LEDA>}.
\end{enumerate}
\end{description}

In case you are working on a different version of MS Windows, please consult the
documentation of your version in order to learn how to perform the corresponding steps.
You might have to restart your computer for the changes to take effect.


\subsection*{Files and Directories}
To compile and link your programs with LEDA, the LEDA main directory should
contain the following files and subdirectories:
\begin{center}
\begin{tabular}{ll}
Readme.txt               & Readme File\\
incl$\backslash$ & the LEDA include directory\\
\end{tabular}
\end{center}
and at least one of the following dll/library sets
\begin{itemize}
\item \texttt{
   leda\_md.dll,
   leda\_md.lib,
   libGeoW\_md.lib}
\item \texttt{
   leda\_mdd.dll,
   leda\_mdd.lib,
   libGeoW\_mdd.lib}
\item \texttt{
   leda\_mt.dll,
   leda\_mt.lib,
   libGeoW\_mt.lib}
\item \texttt{
   leda\_mtd.dll,
   leda\_mtd.lib,
   libGeoW\_mtd.lib}
\end{itemize}

{\bf Note:} A DLL of GeoWin is currently not available.
                                 
\subsection*{Compiling and Linking in Microsoft Visual \CC\ .NET}
\ \\
	We now explain how to proceed in order to compile and link an application program
	using LEDA with MS Visual Studio 2017. If you are using a different version of
	MS Visual Studio, please read and understand the guidelines below and consult the
	documentation of your version of the Studio in order to learn how to perform the
	corresponding steps.

\begin{description}
\item[(1)] In the "File" menu of Visual \CC\ .NET click on "New-->Project".
\item[(2)] Choose "Visual C++" as \texttt{project type} and choose "Empty Project".
\item[(3)] Enter a project name, choose a directory for the project, and click "OK".
\item[(4)] After clicking "OK" you have an empty project space. Choose, for instance,
"Debug" and "x64" (or "x86" in case you are working on a 32-bit system) in the
corresponding pick lists.
\\
\item[If you already have a source file \texttt{prog.cpp}:]
\item[(5)] Activate the file browser and add \texttt{prog.cpp} to the main
 folder  of your project
\item[(6)] In the \texttt{Solution Explorer} of your project click on 
"Source Files" with the right mouse button, then click on 
"Add--> Add Existing Item" with the left mouse button
\item[(7)] Double click on \texttt{prog.cpp}
\\
\item[If you want to enter a new source file:] 
\item[(5')] In the \texttt{Solution Explorer} of your project click on 
"Source Files" with the right mouse button, then click on 
"Add--> Add New Item" with the left mouse button.
\item[(6')] Choose "C++ File" in \texttt{Templates}, enter a name, 
and click "Add".
\item[(7')] Enter your code.
\\
\item[(8)] In the \texttt{Solution Explorer} right click on your project
and left click on "Properties"
\item[(9a)] Click on  "C/\CC" and "Code Generation" and choose 
the "Run Time Library" (=compiler flag) you want to use.

 If you chose "Debug" in step 4, the default value is now "/MDd", alternatives are
      "/MD", "/MT", and "/MTd". 
      Notice that you have to use the LEDA libraries that correspond to the 
      chosen flag, e.g., with option "/MDd" you must use 
      \texttt{libleda\_mdd.lib} and \texttt{libgeow\_mdd.lib}.
      Using another set of libraries with "/MDd" could lead to serious 
      linker errors.
\item[(9b)] Click on  "C/\CC" and "Preprocessor" and add
\texttt{/D "LEDA\_DLL"} in "Preprocessor Definitions".
\item[(10)] Click on "Linker" and "Command Line" and add the name of the 
LEDA libraries you want to
       use in "Additional Options" as follows.
       We use \texttt{<opt>} to indicate the compiler option chosen in 
Step (9) 
       (e.g., \texttt{<opt>} is \texttt{mdd} for "/MDd").
\begin{itemize}   
          \item \texttt{leda\_$<$opt$>$.lib}  \\
            for programs that do not use GeoWin

          \item \texttt{libGeoW\_$<$opt$>$.lib leda\_$<$opt$>$.lib}\\
            for programs using GeoWin 
\end{itemize}
      Alternatively, you can include \texttt{<LEDA/msc/autolink\_dll.h>} in
      your program and the correct LEDA libraries are linked to
      your program automatically. If GeoWin is used you need to
      add \texttt{"\_LINK\_GeoW"}
 to the "Preprocessor definitions" in Step (9).                        
\item[(11)] Click on "V\CC\ Directories" of the "Properties" window.
\item[(12)] Choose "Include Files" and add the 
       directory \texttt{$<$LEDA$>$$\backslash$incl} containing the LEDA 
include files 
       (Click on the line starting with "Include Files", then click on "Edit..." in
	   the pick list at the right end of that line. Push the "New line" button and then enter 
       \texttt{$<$LEDA$>$$\backslash$incl}, or click on the small grey 
       rectangle on the right and
       choose the correct directory.) Alternatively you can click C/\CC--> General in the 
	   Configuration Properties and then edit the line "Additional Include Directories".
\item[(13)] Choose "Library Directories" and add the 
       directory \texttt{$<$LEDA$>$} containing the LEDA libraries.
\item[(14)] Click "OK" to leave the "Properties" 
\item[(15)] In the "Build" menu click on \texttt{"<Build Project>"} or 
\texttt{"Rebuild <Project>"} to compile your program.
\item[(16)] To execute the program "prog.exe" Windows needs to have 
\texttt{leda\_<opt>.dll} in its search path for DLL's. Therefore, you need
to do one of the following.
\begin{itemize}
\item Copy \texttt{leda\_<opt>.dll} to the \texttt{bin$\backslash$} 
subdirectory of your compiler or the directory containing "prog.exe".
\item Alternatively, you can set the environment variable \texttt{PATH}
to the directory containing \texttt{leda\_<opt>.dll} as described below.
\end{itemize}
\item[(17)] In order to execute your program, click the green play button in the tool bar.
\end{description}

   Remark: If your \CC\ source code files has extension .c, you need to add 
            the option "/TP" in "Project Options" (similar to Step (9)), 
            otherwise you will get a number of compiler errors. 
(Click on  "C/\CC" and "Command Line". Add \texttt{/TP} in 
"Additional Options" and click "Apply".)

 If you chose "Debug" for your project type, the default value is "/MDd", 
 alternatives are "/MD", "/MT", and "/MTd". 
      Notice that you have to use the LEDA libraries that correspond to the 
      chosen flag, e.g., with option "/MDd" you must use 
      \texttt{leda\_mdd.lib} and \texttt{libGeoW\_mdd.lib}.
      Using another set of libraries with "/MDd" could lead to serious 
      linker errors.


To add LEDA to an existing Project in Microsoft Visual \CC\ .NET, start the 
Microsoft Visual Studio with your project and follow Steps (8)--(14) above.


\subsection*{Compiling and Linking Application Programs in a DOS-Box}
\begin{description}
\item[(a) Setting the Environment Variables for Visual \CC\ .NET:]
\ \\
The compiler \texttt{CL.EXE} and the linker \texttt{LINK.EXE} require 
that the environment variables \texttt{PATH, INCLUDE}, and \texttt{LIB} have 
been set properly. This can easily be ensured by using the command prompts that are
installed on your computer with your Visual Studio installation.\\
  
      To compile programs together with LEDA, the environment 
      variables \texttt{PATH}, \texttt{LIB}, and \texttt{INCLUDE} must additionally contain the 
      corresponding LEDA directories. We now explain how to do that with MS Windows 10.
	  If you are using a different version of MS Windows, please read and understand the
	  guidelines below and consult the documentation of your operating system in order to
	  learn how to perform the corresponding steps.
 
\item[(b) Setting Environment Variables for LEDA:]
\
\begin{description}
\item[MS Windows 10:]\
\begin{enumerate}
\item Open the Start Search, type in �env�, and choose �Edit the system 
environment variables�. A window titled "System Properties" should open.\\
\ \\
\item  Click the button "Environment variables..." in the lower right corner of the
"System Properties" window. A new window opens that allows to add/change/delete the user
variables for your account as well as the system variables, provided you have admin
rights. If not, change the environment variables of your account. 

             If a user variable \texttt{PATH}, \texttt{LIB}, or
\texttt{INCLUDE} already exists, extend the current value as follows:

\begin{itemize}
             \item extend \texttt{PATH} by \texttt{<LEDA>}
             \item extend \texttt{INCLUDE} by \texttt{<LEDA>$\backslash$incl}
             \item extend \texttt{LIB} by \texttt{<LEDA>}
\end{itemize}

             Otherwise add a new user variable \texttt{PATH, INCLUDE}, or
\texttt{LIB} with value \texttt{<LEDA>}, respectively 
\texttt{<LEDA>$\backslash$incl}.
\end{enumerate}
\end{description}

      You might have to restart your computer for the changes to take effect.


\item[(c) Compiling and Linking Application Programs:]
\ \\
After setting the environment variables, you can use the LEDA libraries 
as follows to compile and link programs.

      Programs that do not use GeoWin:
\begin{verbatim}
      cl <option> -DLEDA_DLL prog.cpp <libleda.lib>
\end{verbatim}
      Programs using GeoWin:
\begin{verbatim}
      cl <option> -DLEDA_DLL prog.cpp <libGeoW.lib> <libleda.lib>
\end{verbatim}
 Possible values for \texttt{<option>} are "-MD", 
"-MDd", "-MT", and "-MTd". You have to use the LEDA libraries that 
correspond to the chosen \texttt{<option>}, e.g., with option "-MD" you 
must use \texttt{leda\_md.lib} and \texttt{libGeoW\_md.lib}.
      Using another set of libraries with "-MD" could lead to serious 
      linker errors.

\end{description}

\subsection*{Example programs and demos}
   The source code of all example and demo programs can be found in 
   the directory \texttt{<LEDA>$\backslash$test} and 
   \texttt{<LEDA>$\backslash$demo}. 
   Goto \texttt{<LEDA>} and type
   \texttt{make\_test} or \texttt{make\_demo} to compile and link all test 
   or demo programs, respectively. 


\section{Namespaces and Interaction with other Libraries} 
\label{NameSpace}

If users want to use other software packages like
STL together with LEDA in one project avoiding naming conflicts
is an issue. 

LEDA defines all names (types, functions,
constants, ...) in the \texttt{namespace leda}. 
This makes the former macro--based prefixing scheme obsolete. 
Note, however, that the prefixed names
leda\_... still can be used for backward compatibility. Application
programs have to use namespace leda globally (by saying \texttt{"using
namespace leda;"}) or must prefix every LEDA symbol with 
\texttt{"leda::"}.

The second issue of interaction concerns the data type \texttt{bool}
which is part of the new \CC{} standard. However not all compilers
currently support a bool type.  LEDA offers \texttt{bool} either 
compiler provided or defined within LEDA if the compiler lacks 
the support. Some STL packages follow a similar scheme. To solve the
existance conflict of two different bool type definitions we suggest
to use LEDA's bool as STL is a pure template library only provided
by header files and its defined bool type can be easily replaced. 


\section{Platforms} \label{Platforms}

Please visit our web pages for information about the supported platforms.

